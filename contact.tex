\documentclass[]{jsarticle}
\usepackage{float}
\usepackage{bm}
\usepackage{amsmath}
\usepackage[dvipdfmx]{graphicx}
\bmdefine{\bex}{x}
\bmdefine{\vex}{v}
\begin{document}
\section{しんちょく}
\subsection{SPH法}
座標$\bm{x}$の粒子がもつ物理量Aは次のように表現される。 \cite{Becker2007}
\begin{equation}
A(\bex)=\sum_j \frac{A_j}{\rho_j} W(\bex-\bex_j,h)
\end{equation}
ここでWはKernel関数と呼ばれるもので、本計算では次の関数を用いた。
\begin{equation}
  q=\frac{|\bex-\bex_j|}{h}, 
  W(\bex-\bex_j,h)=\begin{cases}
    \frac{15}{14\pi h^2} \{(2-q)^3-4(1-q)^3\} & \text{$ 0 \leq q \leq 1$}\\
    \frac{15}{14\pi h^2} (2-q)^3 & \text{$1\leq q \leq 2$}\\
    0 & \text{otherwise}
\end{cases}
\end{equation}
密度は次のように表される。
\begin{equation}
\rho_{a}=\sum_bm_bW_{ab}
\end{equation}
ここで$W_{ab}=W(\bex_a-\bex_b)$である。

得られた密度から圧力を計算する。非圧縮性を満たすためにはPoisson方程式$\nabla^2P=\rho\frac{\nabla\cdot\bm{v}}{\delta t}$を解かなければならないが、計算コストが大きい。そのため本計算ではTait方程式と呼ばれる状態方程式を用いて圧力を導出する。

Tait方程式は次のように表される。
\begin{equation}
P=B((\frac{\rho}{\rho_0})^\gamma-1)
\end{equation}
ここで係数Bは次のように与えられる。
\begin{equation}
B=\frac{\rho_oc_s^2}{\gamma}
\end{equation}
本計算では$\gamma=7$を用いた。
圧力により粒子に与えられる加速度は次のようになる。
\begin{equation}
\frac{d\vex_a}{dt}=-\sum_bm_b(\frac{P_a}{\rho_a^2}+\frac{P_b}{\rho_b^2})\nabla_aW_{ab}
\end{equation}
粘性による加速度はMonaghanによる形式 \cite{Monaghan2005}を用いた。
\begin{equation}
\frac{d\vex_a}{dt}=\begin{cases}
  -\sum_bm_b\Pi_{ab}\nabla_aW_{ab} & \vex_{ab}\cdot\bex_{ab}<0 \\
  0 & \vex_{ab}\cdot\bex_{ab}\geq0
\end{cases}
\end{equation}
ここで$\Pi_{ab}$は以下のように与えられる。
\begin{equation}
\Pi_{ab}=-\nu(\frac{\vex_{ab}\cdot\bex_{ab}}{|\bex_{ab}|^2+\epsilon h^2})
\end{equation}
粘性係数$\nu=\frac{2\alpha h c_s}{\rho_a+\rho_b}$である。

\subsection{近傍粒子の探索法}
粒子法では、ある粒子はその影響範囲内に存在する他の粒子とのみ相互作用を計算すればよい。影響範囲内に存在するか否かを判定する最も単純な方法は、2つの粒子間の距離を計算することであるが、全ての粒子に対して距離の計算を行うと
計算コストが大きくなる。本計算では、バケット構造を用いてある粒子の近傍にある粒子とのみ相互作用の計算をするようにしている。\cite{Koshiduka2014}
 
計算領域全体に正方形のバケットを敷き詰め、それぞれのバケットは次のような番号を持っている
\begin{equation}
ib=ix+iy*nBx
\end{equation}
ここでibはバケットに付与された数、nBxはx軸方向に並べられたバケットの数であり、ix,iyはそれぞれバケットのx,y座標であり、次のようにして求められる。
\begin{equation}
ix=\lfloor(r_x-MIN\_X)/DB \rfloor
\end{equation}
\begin{equation}
iy=\lfloor(r_y-MIN\_Y)/DB \rfloor
\end{equation}
ここで$r_x,r_y$は粒子のxおよびy座標、DBはバケット一辺の長さであり、MIN\_X,MIN\_Yは計算範囲におけるxおよびyの最小値である。

近傍粒子の探索は以下のようにして行う。
\label{subsec:label}


\subsection{時間発展}
本計算の時間発展にはleap-frog法を用いた。Leap-frog法では時刻iにて次のように速度および位置を更新する。
\begin{equation}
  \vex^{i+\frac{1}{2}}=\vex^{i-\frac{1}{2}}+\bm{a}^idt
\end{equation}
\begin{equation}
  \bex^{i+1}=\bex^i+\vex^{i+\frac{1}{2}}dt
\end{equation}
\begin{equation}
  \vex^{i+1}=\vex^{i+\frac{1}{2}}+\frac{\bm{a}^i}{2}dt
\end{equation}
時刻1において$\vex^{-\frac{1}{2}}$が存在しないため、次のようにして導出する。
\begin{equation}
  \vex^{\frac{1}{2}}=\vex^{0}+\frac{\bm{a}^0}{2}dt
\end{equation}
\begin{equation}
  \bex^{1}=\bex^0+\vex^{\frac{1}{2}}dt
\end{equation}
\begin{equation}
  \vex^{1}=\vex^{0}+\bm{a}^idt
\end{equation}


\subsection{表面張力、親水性および疎水性の表現}
表面張力および親水性、疎水性の表現にはYangらの手法 \cite{Yang2017}を用いた。この手法は2粒子の間で力を働かせる手法であり、次の式で表される。
\begin{equation}
F^{\rm{int}}_i=\sum_{j} \bm{f}_{ji}
\end{equation}
\begin{equation}
  \bm{f}_{ji}=\begin{cases}
    c_{ij}m_im_j\cos(\frac{3\pi}{2kh}r_{ij})\frac{\bm{r}_{ij}}{r_{ij}} & \text{$|\bm{r}_{ij}|\leq kh$}\\
    0 & \text{$|\bm{r}_{ij}|\geq kh$}
\end{cases}
\end{equation}
ここで$\bm{r}_{ij}=\bm{r}_i-\bm{r}_{j}$、$r_{ij}=|\bm{r}_{ij}|$である。kはenlargement ratioと呼ばれるパラメータであり、相互作用の力が働く粒子の距離を調節している。

$c_{ij}$は2粒子間の相互作用の強さであり、各粒子がどの相に属しているかによって値は変化する。それらの値は次の関係を満たす。
\begin{equation}
\cos\theta=\frac{c_{\alpha\alpha}-c_{\beta\beta}+2c_{\omega\alpha}-2c_{\omega\beta}}{c_{\alpha\alpha} +c_{\beta\beta} -2c_{\alpha\beta}}
\end{equation}
ここで$\theta$は流体と床の接触角であり、$\alpha,\beta,\omega$はそれぞれ流体、空気、床に粒子が存在していることを意味している。

今、$c_{\alpha\beta},c_{\omega\beta}$は小さいことが物理的な観測より分かっているため0とする。また$c_{\alpha\alpha}=c_{\beta\beta}$としてもよいため
\begin{equation}
\cos\theta\simeq c_{\omega\alpha}/c_{\alpha\alpha}
\end{equation}
となる。
\begin{table}[h]
  \caption{相互作用の計算に用いられるパラメータ}
  \label{interParam}
  \begin{center}
    \begin{tabular}{|c|c|}\hline
      k&3.0 \\ \hline
      $c_{\alpha\alpha}$ & 60.0 \\ \hline
    \end{tabular}
  \end{center}  
\end{table}

半円状の流体を床の上に配置し、流体の運動が平衡状態に達した時の様子を比較する。図\ref{fig:Phoby}\ref{fig:Phile}は接触角として$3\pi/4, \pi/4$を与えた時の流体の振る舞いを、80000ステップ時点で出力したものである。流体との接触角が$\pi/2$より小さいを親水性といい、$\pi/2$よりも大きい物を疎水性という \cite{Truscott2012}。図を見ると接触角がそれぞれ$\pi/2$よりも大きいあるいは小さいことが観測されるため、本モデルは親水性および疎水性の振る舞いを定性的に再現していると考えられる。
\begin{figure}[H]
  \centering
  \includegraphics[width=\columnwidth]{initial.png}
  \caption{初期状態}
  \label{fig:initial}
\end{figure}
\begin{figure}[H]
    \centering
    \includegraphics[width=\columnwidth]{Phoby.png}
    \caption{疎水性}
    \label{fig:Phoby}
  \end{figure}
  \begin{figure}[H]
    \centering
    \includegraphics[width=\columnwidth]{Phile.png}
    \caption{親水性}
    \label{fig:Phile}
\end{figure}

\bibliography{thesis}
\bibliographystyle{junsrt}

\end{document}



