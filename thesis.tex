\documentclass[]{jsarticle}
\usepackage{float}
\usepackage{bm}
\usepackage{amsmath}
\bmdefine{\bex}{x}
\bmdefine{\vex}{v}
\usepackage[dvipdfmx]{graphicx}

\begin{document}
\title{水面に剛体が衝突した際に生じる気泡の解析}
\author{京都大学理学研究科 物理学宇宙物理学専攻\\流体物理学研究室\\佐藤道矩}
\maketitle
\newpage

\tableofcontents
\newpage
\section{イントロダクション}
\subsection{界面の変形}
水面に水滴あるいは剛体を衝突させた際の現象、即ち水面の変形や気泡の生成は身近な物理現象の一つであり、古くから研究が行われてきた。

水面の変形に関する研究では、Worthingtonによるもの \cite{Worthington1908}がその嚆矢として挙げられることが多い。彼は水面に水滴を衝突させる実験を複数回行い、その度に少しずつ異なるタイミングでストロボを焚いて写真を撮影することで、水飛沫の生成の連続写真を作成することに成功した。以来、撮影器具の進歩に合わせ、より鮮明な画像が撮影され続けてきた。水しぶきの画像は、高速度撮影の代表的な成果の一つであり、1957年Harold Edgertonが撮影したミルククラウンの写真は、TIMEが選んだ”最も影響ある写真100枚”に選ばれている。

水面に水滴や剛体を衝突させた際に界面が変形し、空気が水中に取り込まれることで気泡(cavity)が生成される現象が確認されている。生成される気泡の大きさは様々であり、剛体球に空気が引きずられるようなもの\cite{Worthington1908}\cite{Mallock1918}、水面の変形によって空気がちぎり取られるようなもの\cite{Oguz1990}などが一例である。

空気が取り込まれる現象は音生成の原因となっている。Minnaert\cite{minnaert1933xvi}が”小川のせせらぎ、瀑布の轟音、海の潮騒についてほとんど分かっていない”と述べたこの現象は、Franz\cite{Franz1959}やPumphrey\cite{Pumphrey1990}らによって研究され、Leightonによる浩瀚な研究書\it{The Acoustic Bubble}\cite{leightonacoustic}にまとめられた。近年では、ハイスピードカメラを用いて、取り込まれた泡が実際に振動している様子が撮影されている\cite{Phillips2018}。

\subsection{SPH法}
流体を粒子の集合として計算する手法は粒子法と呼ばれ、\rm{Smoothed Particle Hydrodynamics}(略称SPH)法は粒子法のモデルの一つである。

SPH法は1977年に、宇宙物理における気体運動の問題を解くための手法として、MonaghanとGingold\cite{Gingold1977}のペア及びLucy\cite{Lucy1977}が別々に発表した。1980年台には宇宙物理に加えて、電磁流体力学や弾性・破壊過程の計算に用いられるようになったが、非圧縮性流体に適応されるようになったのは1990年代の半ばになってからである。1994年にMonaghan\cite{Monaghan1994}が発表した手法が現在流体計算に用いられているモデルの基本となり、以来計算の目的に応じて様々な改良が施され続けている。

粒子法は流体を物理量を持った粒子の集合として計算する手法であり、計算領域を格子に分割して計算する手法との差を強調してmesh-freeな方法と呼ばれることもある。利点としては、粒子の運動がそのまま流体の運動になるため、流体の分離や合体といった複雑な変形を容易に扱うことができる。
プログラムの記述が容易。
ポワソン方程式を解かない場合、計算時間を大きく削減することが可能。
などが挙げられる。
CG分野でもしばしば用いられており、、、

\label{subsec:label}

\section{SPH法}
\subsection{SPH法の支配方程式}
座標$\bm{x}$の粒子がもつ物理量Aは次のように表現される。 \cite{Becker2007}
\begin{equation}
A(\bex)=\sum_j \frac{A_j}{\rho_j} W(\bex-\bex_j,h)
\end{equation}
ここでWはKernel関数と呼ばれるもので、本計算では次の関数を用いた。
\begin{equation}
  q=\frac{|\bex-\bex_j|}{h}, 
  W(\bex-\bex_j,h)=\begin{cases}
    \frac{15}{14\pi h^2} \{(2-q)^3-4(1-q)^3\} & \text{$ 0 \leq q \leq 1$}\\
    \frac{15}{14\pi h^2} (2-q)^3 & \text{$1\leq q \leq 2$}\\
    0 & \text{otherwise}
\end{cases}
\end{equation}
密度は次のように計算される。
\begin{equation}
\rho_{a}=\sum_bm_bW_{ab}
\end{equation}
ここで$W_{ab}=W(\bex_a-\bex_b)$である。

得られた密度から圧力を計算する。非圧縮性を満たすためにはPoisson方程式$\nabla^2P=\rho\frac{\nabla\cdot\bm{v}}{\Delta t}$を解かなければならないが、計算コストが大きい。そのため本計算ではTait方程式と呼ばれる状態方程式を用いて圧力を導出する。

Tait方程式は次のように表される。
\begin{equation}
P=B((\frac{\rho}{\rho_0})^\gamma-1)
\end{equation}
ここで係数Bは次のように与えられる。
\begin{equation}
B=\frac{\rho_oc_s^2}{\gamma}
\end{equation}
本計算では$\gamma=7$を用いた。
圧力により粒子に与えられる加速度は次のようになる。
\begin{equation}
\frac{d\vex_a}{dt}=-\sum_bm_b(\frac{P_a}{\rho_a^2}+\frac{P_b}{\rho_b^2})\nabla_aW_{ab}
\end{equation}
粘性による加速度はMonaghanによる形式 \cite{Monaghan2005}を用いた。
\begin{equation}
\frac{d\vex_a}{dt}=\begin{cases}
  -\sum_bm_b\Pi_{ab}\nabla_aW_{ab} & \vex_{ab}\cdot\bex_{ab}<0 \\
  0 & \vex_{ab}\cdot\bex_{ab}\geq0
\end{cases}
\end{equation}
ここで$\Pi_{ab}$は以下のように与えられる。
\begin{equation}
\Pi_{ab}=-\nu(\frac{\vex_{ab}\cdot\bex_{ab}}{|\bex_{ab}|^2+\epsilon h^2})
\end{equation}
粘性係数$\nu=\frac{2\alpha h c_s}{\rho_a+\rho_b}$である。

\subsection{剛体の表現}
剛体の表現はPMS(Passively Moving Solid)法\cite{Gotoh2018}と呼ばれるモデルを用いた。
剛体は複数の剛体粒子が集まったものとして表現される。各タイムステップごとに剛体粒子を流体粒子と同様に計算し、速度を導出する。次に得られた速度を用いて以下の量を計算する。
\begin{equation}
  \bm{r}_g=\sum_{i}^n{\bm{r}_i}
\end{equation}
\begin{equation}
  \bm{q}_i=\bm{r}_i-\bm{r}_g
\end{equation}
\begin{equation}
  I=\sum_{i}^n{|\bm{q}_i|^2}
\end{equation}
ここで$\bm{r}_g,\bm{q}_i,I$はそれぞれ重心座標、重心から剛体粒子iへのベクトル、慣性モーメントであり、nは剛体粒子の個数である。これらの値を用いて剛体の並進速度$\bm{T}$および角速度$\bm{R}$を求める。
\begin{equation}
  \bm{T}=\frac{1}{n}\sum_{i}^n{\bm{v}_i}
\end{equation}
\begin{equation}
  \bm{R}=\frac{1}{I}\sum_{i}^n{\bm{v}_i\times\bm{q}_i}
\end{equation}
これらの値から剛体の速度は
\begin{equation}
\bm{v}_i=\bm{T}+\bm{q}_i\times\bm{R}
\end{equation}
となる。ここで新しく得られた速度を用いて剛体粒子の座標を更新する。

\subsection{近傍粒子の探索法}
粒子法では、ある粒子はその影響範囲内に存在する他の粒子とのみ相互作用を計算すればよい。影響範囲内に存在するか否かを判定する最も単純な方法は、2つの粒子間の距離を計算することであるが、全ての粒子に対して距離の計算を行うと
計算コストが大きくなる。本計算では、バケット構造を用いてある粒子の近傍にある粒子とのみ相互作用の計算をするようにしている。\cite{Koshiduka2014}ここで格子は粒子の位置の判別に用いられるのであって、Euler法のように格子自体が物理量を持っているわけではない。
 
計算領域全体に正方形のバケットを敷き詰め、それぞれのバケットは次のような番号を持っている
\begin{equation}
ib=ix+iy*nBx
\end{equation}
ここでibはバケットに付与された数、nBxはx軸方向に並べられたバケットの数であり、ix,iyはそれぞれバケットのx,y座標であり、次のようにして求められる。
\begin{equation}
ix=\lfloor(r_x-MIN\_X)/DB \rfloor
\end{equation}
\begin{equation}
iy=\lfloor(r_y-MIN\_Y)/DB \rfloor
\end{equation}
ここで$r_x,r_y$は粒子のxおよびy座標、DBはバケット一辺の長さであり、MIN\_X,MIN\_Yは計算範囲におけるxおよびyの最小値である。

近傍粒子の探索は以下のようにして行う。


\subsection{時間発展}
本計算の時間発展にはleap-frog法を用いた。Leap-frog法では時刻iにて次のように速度および位置を更新する。
\begin{equation}
  \vex^{i+\frac{1}{2}}=\vex^{i-\frac{1}{2}}+\bm{a}^idt
\end{equation}
\begin{equation}
  \bex^{i+1}=\bex^i+\vex^{i+\frac{1}{2}}dt
\end{equation}
\begin{equation}
  \vex^{i+1}=\vex^{i+\frac{1}{2}}+\frac{\bm{a}^i}{2}dt
\end{equation}
時刻1において$\vex^{-\frac{1}{2}}$が存在しないため、次のようにして導出する。
\begin{equation}
  \vex^{\frac{1}{2}}=\vex^{0}+\frac{\bm{a}^0}{2}dt
\end{equation}
\begin{equation}
  \bex^{1}=\bex^0+\vex^{\frac{1}{2}}dt
\end{equation}
\begin{equation}
  \vex^{1}=\vex^{0}+\bm{a}^0dt
\end{equation}

\subsection{単位系}
剛体球の直径Dと重力加速度gから、単位時間$T_g=(\frac{D}{g})^{1/2}$を定義する。剛体球の初速に$T_g$を乗じたものをDで割ることによって、単位時間あたりに剛体球の直径の何倍移動するかという無次元量で剛体球の初速を定義する。この値を$U_0^*$と書き表す。

例として、実験での各値が、$D=2.0\times10^{-2}m,v_0=5m/s$であるとき、$U_0^*\sim11.25$である。

粒子法での各値は$D=1, g=9.8$であるので、$U_0^*=11.25$を得るために、剛体球の初速は35.12でなければならない。

以上のようにして実験での値と数値計算での値を比較する。




\subsection{表面張力、親水性および疎水性の表現}
表面張力および親水性、疎水性の表現にはYangらの手法 \cite{Yang2017}を用いた。この手法は2粒子の間で力を働かせる手法であり、次の式で表される。
\begin{equation}
F^{\rm{int}}_i=\sum_{j} \bm{f}_{ji}
\end{equation}
\begin{equation}
  \bm{f}_{ji}=\begin{cases}
    c_{ij}m_im_j\cos(\frac{3\pi}{2kh}r_{ij})\frac{\bm{r}_{ij}}{r_{ij}} & \text{$|\bm{r}_{ij}|\leq kh$}\\
    0 & \text{$|\bm{r}_{ij}|\geq kh$}
\end{cases}
\end{equation}
ここで$\bm{r}_{ij}=\bm{r}_i-\bm{r}_{j}$、$r_{ij}=|\bm{r}_{ij}|$である。kはenlargement ratioと呼ばれるパラメータであり、相互作用の力が働く粒子の距離を調節している。

$c_{ij}$は2粒子間の相互作用の強さであり、各粒子がどの相に属しているかによって値は変化する。それらの値は次の関係を満たす。
\begin{equation}
\cos\theta=\frac{c_{\alpha\alpha}-c_{\beta\beta}+2c_{\omega\alpha}-2c_{\omega\beta}}{c_{\alpha\alpha} +c_{\beta\beta} -2c_{\alpha\beta}}
\end{equation}
ここで$\theta$は流体と床の接触角であり、$\alpha,\beta,\omega$はそれぞれ流体、空気、床に粒子が存在していることを意味している。

今、$c_{\alpha\beta},c_{\omega\beta}$は小さいことが物理的な観測より分かっているため0とする。また$c_{\alpha\alpha}=c_{\beta\beta}$としてもよいため
\begin{equation}
\cos\theta\simeq c_{\omega\alpha}/c_{\alpha\alpha}
\end{equation}
となる。
\begin{table}[h]
  \caption{相互作用の計算に用いられるパラメータ}
  \label{interParam}
  \begin{center}
    \begin{tabular}{|c|c|}\hline
      k&3.0 \\ \hline
      $c_{\alpha\alpha}$ & 60.0 \\ \hline
    \end{tabular}
  \end{center}  
\end{table}

\newpage
\section{計算結果}
\subsection{粘性、表面張力の測定}
粒子法における粘性・表面張力と、水が持つ粘性・表面張力を比較するために、次の数値計算を行った。

\label{subsec:label}


\subsection{親水性、疎水性の妥当性}
\label{subsec:label}


半円状の流体を床の上に配置し、流体の運動が平衡状態に達した時の様子を比較する。図\ref{fig:Phoby}\ref{fig:Phile}は接触角として$3\pi/4, \pi/4$を与えた時の流体の振る舞いを、80000ステップ時点で出力したものである。流体との接触角が$\pi/2$より小さいを親水性といい、$\pi/2$よりも大きい物を疎水性という \cite{Truscott2012}。図を見ると接触角がそれぞれ$\pi/2$よりも大きいあるいは小さいことが観測されるため、本モデルは親水性および疎水性の振る舞いを定性的に再現していると考えられる。
\begin{figure}[H]
  \centering
  \includegraphics[width=\columnwidth]{initial.png}
  \caption{初期状態}
  \label{fig:initial}
\end{figure}
\begin{figure}[H]
    \centering
    \includegraphics[width=\columnwidth]{Phoby.png}
    \caption{疎水性}
    \label{fig:Phoby}
  \end{figure}
  \begin{figure}[H]
    \centering
    \includegraphics[width=\columnwidth]{Phile.png}
    \caption{親水性}
    \label{fig:Phile}
\end{figure}

\subsection{剛体球の水面への衝突}
表面が親水性および疎水性の剛体球を水面に衝突させる。初速を変化させて、水面の変形による気泡の取り込みが見られるか否かを調べる。実験の値はDuez\cite{Duez2007}らによるものを参照した。
\label{subsec:label}

\newpage

\section{議論}
The cavity! The cavity!
\newpage
\section{結論}
\subsection{まとめ}
Blah, blah, blah...
\newpage

\section{謝辞}
Thank, danke, merci!
\newpage

\bibliography{thesis}
\bibliographystyle{junsrt}


\end{document}
