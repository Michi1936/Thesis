\documentclass[]{jsarticle}
\usepackage{float}
\usepackage{bm}
\usepackage{amsmath}
\bmdefine{\bex}{x}
\bmdefine{\vex}{v}
\usepackage[dvipdfmx]{graphicx}
\usepackage{subcaption}

\begin{document}
\title{親水性および疎水性の剛体を水面に衝突させた際に生じる空洞の数値解析}
\author{京都大学理学研究科 物理学宇宙物理学専攻\\流体物理学研究室\\佐藤道矩}
\maketitle
\newpage

\tableofcontents
\newpage
\section{イントロダクション}
\subsection{界面の変形}
水面に水滴あるいは剛体を衝突させた際の現象、即ち水面の変形や気泡の生成は身近な物理現象の一つであり、古くから研究が行われてきた。

水面の変形に関する研究では、Worthingtonによるもの \cite{Worthington1908}がその嚆矢として挙げられることが多い。彼は水面に水滴を衝突させる実験を複数回行い、その度に少しずつ異なるタイミングでストロボを焚いて写真を撮影することで、水飛沫の生成の連続写真を作成することに成功した。以来、撮影器具の進歩に合わせ、より鮮明な画像が撮影され続けてきた。

水面に水滴や剛体を衝突させた際に界面が変形し、空洞(cavity)が生じる現象が確認されている。大気中で実験を行った場合、生成される空洞は気泡と呼ばれる。気泡の大きさや生成される位置は様々であり、Aristoffらは空洞が閉じる位置に応じて、shallow sealやdeep sealといった分類を行っている\cite{Aristoff2009}。これら空洞の分類については、Truscottらによるreview論文\cite{Truscott2014}の模式図が視覚的に分かりやすい。

水面に水滴や個体が衝突した際に音が生じるが、この音生成にも空洞の生成は関係している。
”小川のせせらぎ、瀑布の轟音、海の潮騒についてほとんど分かっていない”とMinnaert\cite{minnaert1933xvi}が述べたこの現象は、Franz\cite{Franz1959}やPumphrey\cite{Pumphrey1990}らによって研究され、Leightonによる浩瀚な研究書\it{The Acoustic Bubble}\rm\cite{leightonacoustic}にまとめられた。近年では、ハイスピードカメラを用いて、取り込まれた気泡が実際に振動している様子が撮影されている\cite{Phillips2018}。

\subsection{SPH法}
界面の変形は流体の分離・合体を伴う現象であり、格子に依存した数値計算法での取り扱いは困難とされる。本論文ではこうした変形を容易に扱える粒子法の一つ、Smoothed Particle Hydrodynamics(略称SPH)法を採用した。

SPH法は1977年に、宇宙物理における気体運動の問題を解くための手法として、MonaghanとGingoldのペア\cite{Gingold1977}及びLucy\cite{Lucy1977}が別々に発表した。1980年代には宇宙物理に加えて、電磁流体力学や弾性・破壊過程の計算に用いられるようになったが、非圧縮性流体に適応されるようになったのは1990年代の半ばになってからである。1994年にMonaghan\cite{Monaghan1994}が発表した手法が現在流体計算に用いられているモデルの基本となり、以来計算の目的に応じて様々な改良が施され続けている(この史観はVioleau\cite{Violeau2016}の論文に依拠している)。

粒子法は流体を物理量を持った粒子の集合として計算する手法であり、計算領域を格子に分割して計算する手法との差を強調してmesh-freeな方法と呼ばれることもある。粒子の運動がそのまま流体の運動になるため、流体の分離や合体といった複雑な変形を容易に扱うことが大きな利点である。

他の利点としては、プログラムの記述が容易であること、圧力計算の際に状態方程式を用いることで計算時間を大きく削減できることなどが挙げられる。

欠点乃至現在の課題として、(i)数値安定性(ii)収束性(iii)境界条件(iv)適応性(adaptivity)、の4つが\cite{Violeau2016}では挙げられている。
また、粒子の大きさに解像度が依存するため、大規模な現象や詳細な解析には大量の粒子が必要となり、計算時間が増大することや、激しい流れにおいてはHigh-Frequency Noiseと呼ばれる揺らぎが生じることも、欠点として挙げられる。

\subsection{先行研究}
剛体球を水面に落とした際の変形に関する研究は、上で述べた通り古くから多数が行われている。ただ球を水面に落下させるだけではなく、剛体球表面の性質を変えたり、水面突入時の剛体に回転運動を与えるなど、様々な発展形が存在している。

本論文では、剛体表面の濡れ性と剛体後流に生成される空洞の関係について注目する。以下、このテーマについての研究史を述べる。

剛体球表面の性質と生成される空洞の関係に対しては、Worthingtonによってすでに指摘されている\cite{Worthington1908}。彼は小石をハンカチで磨く、あるいはサンドペーパーでこすることで表面の性質を変化させた。前者の性質はsmooth、後者の性質はroughと形容され、これらの球が水面に衝突した際にできる水しぶきや空洞が大きく異なることを報告している。

Duez\cite{Duez2007}らによる研究は、親水性あるいは疎水性の剛体球を水面に衝突させ、空洞が生じる衝突速度のしきい値を各接触角について実験から導いた。図\ref{fig:DuezRef}は接触角としきい値の関係図である。また、彼らは、剛体速度と濡れの関係に関するEggers\cite{Eggers2004}の議論を基に、衝突速度のしきい値についての理論式を導出している。剛体速度と濡れの関係の議論はLandau\cite{landau1942}らによるものまで遡る。
\begin{figure}[H]
  \centering
  \includegraphics[width=0.5\columnwidth]{DuezRef.png}
  \caption{Duez\cite{Duez2007}より転載。横軸に接触角を縦軸に空洞が生じる速度のしきい値を取っている。\label{fig:DuezRef} }
\end{figure}


SPH法における濡れ性のモデルは次のものがある。

一つは粒子間に相互作用力を導入して表面張力や濡れ性を再現するモデルであり、Pair-wise Forceモデルと呼ばれる\cite{Tartakovsky2005}。この手法は、表面張力などの力が微視的には分子間力によるものであることから着想を得ている。
粒子間の相互作用力の取り方は多数考案されている\cite{Tartakovsky2005},\cite{Becker2007},\cite{Akinci2013}。
しかし不自然な網目模様が現れたり、相互作用力以外の補正項が必要になる\cite{Akinci2013}など、粒子間の相互作用のみで表面張力や濡れ性を表すのは困難であった。Yang\cite{Yang2016},\cite{Yang2017}らの手法は、より広い範囲の粒子からの相互作用を考えることで、表面張力や濡れ性を特別な補正無しで実現した。

本論文の目的は、剛体が水面に衝突した際に生じる空洞に対し、剛体表面の親水・疎水性の違いが及ぼす影響を、SPH法を用いた2次元の数値計算を通して考察することである。

\newpage
\section{SPH法}
\subsection{SPH法の支配方程式}
\subsubsection{Kernel関数による補間}
座標$\bm{x}$の粒子がもつ物理量Aは次のように表現される\cite{Monaghan2005}、
\begin{equation}  
  \label{eqn:SPH_int}
A(\bex)=\sum_b \frac{A_b}{\rho_b} W(\bex-\bex_b,h).
\end{equation}
ここで$A_b=A(\bex_b)$である。WはKernel関数と呼ばれるもので、本計算では次の関数を用いた、
\begin{equation}
  \begin{split}
    &q=\frac{|\bex-\bex_b|}{h}, \\
    &W(\bex-\bex_b,h)=\begin{cases}
      \frac{15}{14\pi h^2} \left\{(2-q)^3-4(1-q)^3 \right\} & \text{$ 0 \leq q \leq 1$,}\\
      \frac{15}{14\pi h^2} (2-q)^3 & \text{$1\leq q \leq 2$,} \\
      0 & \text{otherwise.}
    \end{cases}
  \end{split}
\end{equation}
式(\ref{eqn:SPH_int})は、ある位置$\bex$に関して、$\bex$周囲に存在する粒子の持つ物理量Aとその粒子までの距離に応じて減衰するKernel関数を掛けあわせた量を足し合わせることで、$\bex$の粒子が持つ物理量Aを求めている。

式(\ref{eqn:SPH_int})はKernel関数が微分可能である場合、両辺を微分することが可能であり、次のように表される、
\begin{equation}
\frac{\partial A}{\partial x}=\sum_bm_b\frac{A_b}{\rho_b}\frac{\partial W}{\partial x}.
\end{equation}

例として密度を計算するには、式(\ref{eqn:SPH_int})の$A_b$を$\rho_b$とすればよく、次のように表される、
\begin{equation}
\rho_{a}=\sum_bm_bW_{ab}.
\end{equation}
ここで$W_{ab}=W(\bex_a-\bex_b)$である。

\subsubsection{圧力項}
得られた密度から圧力を計算する。非圧縮性を満たすためにはPoisson方程式$\frac{\nabla^2p}{\rho}=\nabla\cdot(\vex\cdot\nabla\vex)+\nabla \cdot \bm{f}$を解かなければならないが、計算コストが大きい。そのため本計算ではTait方程式と呼ばれる状態方程式を用いて圧力を導出する。

Tait方程式は次のように表される、
\begin{equation}
  \label{eqn:Tait}
  P=B\left(
 \left(\frac{\rho}{\rho_0}\right)^\gamma-1
    \right).
\end{equation}
本計算では$\gamma=7$である。

式(\ref{eqn:Tait})で得られた圧力を用いて、粒子に与えられる加速度は次のように与えられる、
\begin{equation}
  \label{eqn:pressure_Ante}
\frac{d\vex_a}{dt}=-\frac{1}{\rho_a}\sum_bm_b\frac{P_a}{\rho_b}\nabla_aW_{ab}.
\end{equation}
しかし、この式では粒子bから粒子aに働く力の大きさと粒子aから粒子bに働く力の大きさが等しくない即ち
\begin{equation}
\frac{m_am_bP_b}{\rho_a\rho_b}\nabla_aW_{ab} \neq -\frac{m_am_bP_a}{\rho_a\rho_b}\nabla_bW_{ab}
\end{equation}
であるため、運動量が正しく保存されない。そこで次のように式を変形する。

\begin{equation}
\frac{\nabla P}{\rho}=\nabla \left(\frac{P}{\rho}\right)+\frac{P}{\rho^2}\nabla \rho
\end{equation}
であることを用いると、式(\ref{eqn:pressure_Ante})は
\begin{equation}
\frac{d\vex_a}{dt}=-\sum_bm_b\left(\frac{P_a}{\rho_a^2}+\frac{P_b}{\rho_b^2}\right)\nabla_aW_{ab}
\end{equation}
と書くことができる。$|\bm{r}_a-\bm{r}_b|$のスカラー関数$F_{ab}$を用いて$\nabla_aW_{ab}=\bm{r}_{ab}F_{ab}$と書くと、粒子bから粒子aに働く力は
\begin{equation}
m_am_b\left(\frac{P_b}{\rho_b^2}+\frac{P_a}{\rho_a^2}\right)\bm{r}_{ab}F_{ab}
\end{equation}
と表すことができる。この式より粒子bから粒子aに働く力と粒子aから粒子bに働く力は、向きが反対で大きさが同じであることが分かる。以上の議論は\cite{Monaghan2005}に依った。

式(\ref{eqn:Tait})における係数Bについて詳述する。

係数Bは密度の変化幅に関係する係数であり、数値計算における音速と関連付けられる。音速は
\begin{equation}
c_s=\sqrt{\frac{\partial p}{\partial \rho}}
\end{equation}
で与えられるため、圧力が式(\ref{eqn:Tait})で与えられるとき
\begin{equation}
c_s^2=\frac{\partial p}{\partial \rho}=\frac{B\gamma}{\rho_0}\left(\frac{\rho}{\rho_0}\right)^{\gamma -1}
\end{equation}
となる。密度$\rho_0$の時の音速を$c_{s0}$と表すと
\begin{equation}
B=\frac{\rho_0c_{s0}^2}{\gamma}
\end{equation}
となり、式(\ref{eqn:Tait})は
\begin{equation}
P=\frac{\rho_0c_{s0}^2}{\gamma}\left(\left(\frac{\rho}{\rho_0}\right)^\gamma-1\right)
\end{equation}
となる。

音速$c_S$の値によって密度の変化幅は制御される。$\Delta \rho=\rho-\rho_0$としたとき、密度の変化$\frac{\left| \Delta \rho \right|}{\rho_0}$は、
\begin{equation}
\frac{\left| \Delta \rho \right|}{\rho_0}\sim \frac{\left| \vex_f \right|^2}{c_s^2}
\end{equation}
で評価することができる。ここで$\vex_f$は取り扱う現象に於ける典型的な流速である。

密度変化の上限を$\eta$として、$\frac{\left| \Delta \rho \right|}{\rho_0}\sim \eta$を得るためには、$\frac{v_f}{c_s}<\sqrt{\eta}$が必要となる。以上の議論を基に、取り扱う現象の流速に応じて$c_s$は適当な値を入力する。係数Bの議論は\cite{Becker2007},\cite{Gotoh2018}に依る。


\subsubsection{人工粘性項}
計算の安定性向上および衝撃波の取り扱いのために人工粘性を導入する。

粘性項を粒子間の相対運動から導出するために、$\nabla_aW_{ab}$にかかる$\frac{P_a}{\rho_a^2}+\frac{P_b}{\rho_b^2}$の部分に新しく$\Pi_{ab}$という項を追加する。人工粘性の圧力を線形な形で表すと
\begin{equation}
  q=\begin{cases}
    -\alpha \rho h c \nabla \cdot v & \nabla\cdot v<0, \\
    0 & \nabla \cdot v>0,
  \end{cases}
\end{equation}
となるため、$\Pi_{ab}$は
\begin{equation}
  -\alpha \frac{hc}{\rho}\frac{\partial v}{\partial x}
\end{equation}
という形が予想される。ここで、
\begin{equation}
  \frac{v_{ab}}{x_{ji}}=\frac{v_a-v_b}{x_{ab}}=\frac{1}{x_{ab}}\left[ v_a-\left( v_a+x_{ji}\frac{\partial v_a}{\partial x_a} \right) \right] \sim \frac{\partial v_a}{\partial x_a}
\end{equation}
であるため、
\begin{equation}
\Pi_{ab}=-\alpha \frac{hc_a}{\rho_a}\frac{v_{ab}}{x_{ab}}
\end{equation}
となる。

この式には扱いづらい箇所が2点存在している。一つは分母の$x_{ab}$が$0$になったとき発散してしまうこと、一つは$\frac{c_a}{\rho_a}$のために$\Pi_{ab}\neq\Pi_{ba}$となっていることである。

分母が0になることを防ぐために$\frac{1}{x_{ab}}$を
\begin{equation}
\frac{x_{ab}}{x_{ab}^2+\epsilon h^2},
\end{equation}
という式で置き換える。ここで$\epsilon$は1より十分小さい定数である。

$c_a$および$\rho_a$はそれぞれ
\begin{equation}
  \begin{split}
  \bar{c_{ab}}=\frac{1}{2}(c_a+c_b), \\
  \bar{\rho_{ab}}=\frac{1}{2}(\rho_a+\rho_b),
  \end{split}
\end{equation}
という量で置き換える。

以上の議論より人工粘性による加速度は次のようになる、
\begin{equation}
\frac{d\vex_a}{dt}=\begin{cases}
  -\sum_bm_b\Pi_{ab}\nabla_aW_{ab} & \vex_{ab}\cdot\bex_{ab}<0, \\
  0 & \vex_{ab}\cdot\bex_{ab}\geq0.
\end{cases}
\end{equation}
\begin{equation}
\Pi_{ab}=-\nu\left(\frac{\vex_{ab}\cdot\bex_{ab}}{|\bex_{ab}|^2+\epsilon h^2}\right).
\end{equation}
$\nu$は人工粘性を制御する係数であり、
\begin{equation}
  \label{eqn:kinArtVisc}
  \nu=\frac{2\alpha h c_s}{\rho_a+\rho_b}
\end{equation}
で決定される。以上の議論は\cite{Monaghan1983}に依る。

\subsection{表面張力、親水性および疎水性の表現}
表面張力および親水性、疎水性の表現にはYangらの手法 \cite{Yang2016},\cite{Yang2017}を用いた。この手法は粒子の間に力を作用させることで、表面張力や個体表面での性質を再現する。粒子$i$に加えられる力は、
\begin{equation}
F^{\rm{int}}_i=\sum_{j} \bm{f}_{ji},
\end{equation}
\begin{equation}
  \bm{f}_{ji}=\begin{cases}
    c_{ij}m_im_j\cos\left(\frac{3\pi}{2kh}r_{ij}\right)\frac{\bm{r}_{ij}}{r_{ij}} & \text{$|\bm{r}_{ij}|\leq kh,$}\\
    0 & \text{$|\bm{r}_{ij}|\geq kh.$}
\end{cases}
\end{equation}
ここで$\bm{r}_{ij}=\bm{r}_i-\bm{r}_{j}$、$r_{ij}=|\bm{r}_{ij}|$である。$k$はenlargement ratioと呼ばれるパラメータであり、相互作用力が働く距離を調節している。

$c_{ij}$は2粒子間の相互作用の強さであり、各粒子がどの相に属しているかによって値は変化する。それらの値は次の関係を満たす。
\begin{equation}
\cos\theta=\frac{c_{\alpha\alpha}-c_{\beta\beta}+2c_{\omega\alpha}-2c_{\omega\beta}}{c_{\alpha\alpha} +c_{\beta\beta} -2c_{\alpha\beta}.}
\end{equation}
ここで$\theta$は流体と床の接触角であり、$\alpha,\beta,\omega$はそれぞれ流体、空気、床に粒子が存在していることを表す。例えば$c_{\alpha\omega}$は流体粒子と個体粒子の間に働く相互作用力の大きさ。

今、$c_{\alpha\beta},c_{\omega\beta}$は小さいことが物理的な観測より分かっているため0とする。また$c_{\alpha\alpha}=c_{\beta\beta}$としてもよいため
\begin{equation}
\label{eqn:contactAngleSimplified}
  \cos\theta\simeq c_{\omega\alpha}/c_{\alpha\alpha}
\end{equation}
となる。

\subsection{壁粒子の表現}
壁となる領域に座標を固定した粒子を配置する方法を採用した。壁粒子が一列である場合、壁近傍での圧力が低下し流体粒子が壁を通り抜けることがある。この現象を防ぐため、3層にわたって壁粒子を配置した。
\begin{figure}[H]
  \centering
  \includegraphics[width=0.6\columnwidth]{wallBoundary.png}
  \caption{壁粒子の様子}
  \label{fig:wallParticles} 
\end{figure}

\subsection{剛体の表現}
剛体の表現はPMS(Passively Moving Solid)法\cite{Gotoh2018}と呼ばれるモデルを用いた。

剛体は複数の剛体粒子が集まったものとして表現される。各タイムステップごとに剛体粒子を流体粒子と同様に計算し、速度を導出する。次に得られた速度を用いて以下の量を計算する。
\begin{equation}
  \bm{r}_g=\sum_{i}^n{\bm{r}_i}
\end{equation}
\begin{equation}
  \bm{q}_i=\bm{r}_i-\bm{r}_g
\end{equation}
\begin{equation}
  I=\sum_{i}^n{|\bm{q}_i|^2}
\end{equation}
ここで$\bm{r}_g,\bm{q}_i,I$はそれぞれ重心座標、重心から剛体粒子iへのベクトル、慣性モーメントであり、nは剛体粒子の個数である。これらの値を用いて剛体の並進速度$\bm{T}$および角速度$\bm{R}$を求める。
\begin{equation}
  \bm{T}=\frac{1}{n}\sum_{i}^n{\bm{v}_i}
\end{equation}
\begin{equation}
  \bm{R}=\frac{1}{I}\sum_{i}^n{\bm{v}_i\times\bm{q}_i}
\end{equation}
これらの値から剛体の速度は
\begin{equation}
\bm{v}_i=\bm{T}+\bm{q}_i\times\bm{R}
\end{equation}
となる。ここで新しく得られた速度を用いて剛体粒子の座標を更新する。

\subsection{近傍粒子の探索法}

粒子法では、ある粒子$a$はその影響半径内に存在する他の粒子からのみ相互作用を受ける。
他の粒子が$a$の影響半径内に存在するか否かを判定する最も単純な方法は、2つの粒子間の距離を計算することである。
しかし粒子$a$から遠く離れた粒子とまで距離の計算を行い、影響半径との比較を行うことは計算時間の増大を招く。

本計算ではバケット構造を用い、粒子$a$近傍に存在している粒子とのみ粒子間距離および相互作用の計算を行っている\cite{Koshiduka2014}。ここで格子は粒子の位置の判別に用いられるのであって、Euler法のように格子自体が物理量を持っているわけではない。
 

計算領域全体に正方形のバケットを敷き詰める。粒子$a$が座標$(r_x, r_y)$に存在している時、粒子$a$の入っているバケットの番号は次のように与えられる、
\begin{equation}
\rm{ib=ix+iy*nBx}.
\end{equation}
ここでibはバケットの番号、nBxはx軸方向に並べられたバケットの数である。ix,iyはそれぞれバケットのx,y座標であり、次のようにして求められる。
\begin{equation}
\label{eqn:backet}
  \begin{split}
  \rm{ix=\lfloor(r_x-MIN\_X)/DB \rfloor}\\
  \rm{iy=\lfloor(r_y-MIN\_Y)/DB \rfloor}
\end{split}
\end{equation}
ここでDBはバケット一辺の長さであり、MIN\_X,MIN\_Yは計算範囲におけるxおよびyの最小値である。

近傍粒子の探索は以下のようにして行う。

粒子$a$の座標からixとiyを求め、バケットの番号ibを得る。また、バケットibの周囲に存在する8つのバケットの番号も得ることができる。バケットibおよびその周囲の8バケット内に存在している粒子と粒子$a$との距離を計算し、相互作用を計算する。


\subsection{時間発展}
本計算の時間発展にはleap-frog法を用いた。leap-frog法では時刻$i$にて次のように速度および位置を更新する。
\begin{equation}
  \vex^{i+\frac{1}{2}}=\vex^{i-\frac{1}{2}}+\bm{a}^idt,
\end{equation}
\begin{equation}
  \bex^{i+1}=\bex^i+\vex^{i+\frac{1}{2}}dt,
\end{equation}
\begin{equation}
  \vex^{i+1}=\vex^{i+\frac{1}{2}}+\frac{\bm{a}^i}{2}dt.
\end{equation}
時刻1において$\vex^{-\frac{1}{2}}$が存在しないため、次のようにして導出する。
\begin{equation}
  \vex^{\frac{1}{2}}=\vex^{0}+\frac{\bm{a}^0}{2}dt,
\end{equation}
\begin{equation}
  \bex^{1}=\bex^0+\vex^{\frac{1}{2}}dt,
\end{equation}
\begin{equation}
  \vex^{1}=\vex^{0}+\bm{a}^0dt.
\end{equation}

\subsection{単位系}
剛体球の直径Dと重力加速度gから、単位時間$T_g=\left(\frac{D}{g}\right)^{1/2}$を定義する。剛体球の初速に$T_g$を乗じたものをDで割ることによって、単位時間あたりに剛体球の直径の何倍移動するかという無次元量で剛体球の初速を定義する。この値を$U_0^*$と書き表す。

例として、実験での各値が、$D=2.0\times10^{-2}$[m],$v_0$=5[m/s]であるとき、$T_g\sim 0.045$となり$U_0^*\sim11.25$である。

粒子法での各値は$D=1, g=9.8$であるので、$T_g\sim 0.32$となり、$U_0^*=11.25$を得るために、剛体球の初速は35.12でなければならない。

以上のようにして実験での値と数値計算での値を比較する。
\begin{table}[h]
  \caption{数値計算の初速と実験での初速の対応}
  \label{tab:velocity_non_dim}
  \begin{center}
    \begin{tabular}{|c|c|c|}\hline
      数値計算での初速&無次元化された初速&実験での初速 \\ \hline
      10.00 & 3.19 & 1.41 \\ \hline
      20.00 & 6.39 & 2.83 \\ \hline
      30.00 & 9.58 & 4.24 \\ \hline
      40.00 & 12.78 & 5.66 \\ \hline
      50.00 & 15.97 & 7.07 \\ \hline
      60.00 & 19.17 & 8.49 \\ \hline
      70.00 & 22.36 & 9.90 \\ \hline
      80.00 & 25.56 & 11.31 \\ \hline
    \end{tabular}
  \end{center}
\end{table}



\newpage
\section{計算結果}
\subsection{表面張力の測定}
\label{subsec:surfaceCoef}
粒子法における表面張力と、水が持つ表面張力を比較するために、次の数値計算を行った。
計算に用いたパラメータは以下のとおりである。
\begin{table}[h]
  \caption{節\ref{subsec:surfaceCoef}における各パラメータ}
  \begin{center}
    \begin{tabular}{|c|c|c|}\hline
      影響半径&$h$&$\rm{1.00\times10^{-1}}$ \\ \hline
      流体の密度&$\rho_0$&$\rm{1.00\times10^3}$ \\ \hline
      粒子の質量&$m$&2.62 \\ \hline
      人口粘性項&$\alpha$&$\rm{2.00\times10^{^2}}$ \\ \hline
      重力加速度&$g$&9.80 \\ \hline
      音速&$c_s$&$\rm{8.00\times10^1}$ \\ \hline
      時間刻み幅&$dt$&$\rm{1.0\times10^{-4}}$ \\ \hline
      enlargement length&$k$&3.00 \\ \hline
    \end{tabular}
  \end{center}
\end{table}

図\ref{fig:waveSurfInit}のように水面を波立たせた初期条件を与え、時間発展をさせる。その時に見られる波の波長を理論値と比べた。
まず重力波の比較を行う。流体粒子間の相互作用力は考えていない。重力波の分散関係は
\begin{equation}
  \label{eqn:dispRel}
  \omega(k)=\sqrt{gk \tanh (kh)}
\end{equation}
で与えられる\cite{tatsumiKiso}。$g=9.8, k=\frac{2\pi}{\lambda}=\frac{2\pi}{6.0}, h=3.0$を代入すると$\omega\sim3.18$である。これより周期は$\frac{2\pi}{3.18}\sim1.98$となる。
図\ref{fig:waveSurfEvol}より、実験で得られた周期はおよそ2.0となる。この値は式(\ref{eqn:dispRel})より得られる値とよく合っており、水面波を大方再現しているものと考えられる。
\begin{figure}[H]
  \centering
  \includegraphics[width=0.6\columnwidth]{waveSurf_1226_1953_3.png}
  \caption{粒子の初期配置}
  \label{fig:waveSurfInit}
\end{figure}
\begin{figure}[H]
  \centering
  \includegraphics[width=0.6\columnwidth]{waveSurf_1226_1953_.png}
  \caption{$x$=12における波高の時間発展}
  \label{fig:waveSurfEvol}
\end{figure}


次に表面張力係数の比較を行う。流体粒子間の相互作用$c_{\alpha\alpha}=30$とした。
\begin{figure}[H]
  \centering
  \includegraphics[width=0.6\columnwidth]{waveSurf_1227_1200.png}
  \caption{表面張力を考慮した際の$x$=12における波高の時間発展}
  \label{fig:waveSurfEvolwithSurfTension}
\end{figure}

表面張力を考慮した水面派の分散関係は
\begin{equation}
\omega=\sqrt{\left(g+\frac{\gamma}{\rho}k^2 \right)k \tanh (kh)}
\end{equation}
で与えられる\cite{tatsumiKiso}。$\rho\sim1000, \omega\sim4.18$の値を用いて$\gamma$の値を求めると、6399.5を得た。この値に関する議論は節\ref{subsec:discSurfWave}にて行う。


\subsection{親水性、疎水性の妥当性}
\label{subsec:validity}

半円状の流体を床の上に配置し、流体の運動が平衡状態に達した時の様子を比較する。流体粒子間の相互作用の強さ$c_{\alpha\alpha}=60$とし、その他のパラメータは節\ref{subsec:surfaceCoef}と同じ値を用いた。
\begin{figure}[H]
  \centering
  \includegraphics[width=0.8\columnwidth]{Phobyhalf40_1116_1906_1.png}
  \caption{初期状態}
  \label{fig:contactInitial}
\end{figure}

\begin{figure}[H]
  \centering
  \begin{subfigure}{0.8\columnwidth}
    \centering
    \includegraphics[width=\columnwidth]{Phobyhalf40_1116_1906_1601.png}
    \caption{疎水性}
    \label{fig:PhobyDrop}
  \end{subfigure}
  \begin{subfigure}{0.8\columnwidth}
    \centering
    \includegraphics[width=\columnwidth]{Philehalf40_1118_1813_1601.png}
    \caption{親水性}
    \label{fig:PhileDrop}
  \end{subfigure}
  \caption{親水性および疎水性の床の上に放置された水滴}
  \label{fig:contactAngles}
\end{figure}

図\ref{fig:PhobyDrop}および\ref{fig:PhileDrop}は接触角として$\frac{3}{4}\pi, \frac{\pi}{4}$を与えた時の流体の振る舞いを、80000ステップ時点で出力したものである。流体との接触角が$\frac{\pi}{2}$より小さいを親水性といい、$\frac{\pi}{2}$よりも大きい物を疎水性という \cite{Law2014}。図を見ると接触角がそれぞれ$\frac{\pi}{2}$よりも大きいあるいは小さいことが観測されるため、本モデルは親水性および疎水性の振る舞いを定性的に再現していると考えられる。

次に接触角の設定を変えて数値計算を行った。
\begin{figure}[H]
  \centering
  \begin{subfigure}{0.45\columnwidth}
    \centering
    \includegraphics[width=\columnwidth]{Phobyhalf40_0121_1734_1001.png}
    \caption{接触角$\frac{7}{12}\pi$}
    \label{fig:phoby_pi3}
  \end{subfigure}
  \begin{subfigure}{0.45\columnwidth}
    \centering
    \includegraphics[width=\columnwidth]{Phobyhalf40_0121_1734_1001_wl.png}
    \caption{補助線付き}
    \label{fig:phoby_pi3wl}
  \end{subfigure}

  \begin{subfigure}{0.45\columnwidth}
    \centering
    \includegraphics[width=\columnwidth]{Phobyhalf40_0121_1740_1001.png}
    \caption{接触角$\frac{3}{4}\pi$}
    \label{fig:phoby_pi4}
  \end{subfigure}
  \begin{subfigure}{0.45\columnwidth}
    \centering
    \includegraphics[width=\columnwidth]{Phobyhalf40_0121_1740_1001_wl.png}
    \caption{補助線付き}
    \label{fig:phoby_pi4wl}
  \end{subfigure}

  \begin{subfigure}{0.45\columnwidth}
    \centering
    \includegraphics[width=\columnwidth]{Phobyhalf40_0121_1746_1001.png}
    \caption{接触角$\frac{11}{12}\pi$}
    \label{fig:phoby_pi6}
  \end{subfigure}
  \begin{subfigure}{0.45\columnwidth}
    \centering
    \includegraphics[width=\columnwidth]{Phobyhalf40_0121_1746_1001_wl.png}
    \caption{補助線付き}
    \label{fig:phoby_pi6wl}
  \end{subfigure}

  \caption{接触角を変化させた疎水性の床上の水滴}
\end{figure}

\begin{figure}[H]
  \centering
  \begin{subfigure}{0.45\columnwidth}
    \centering
    \includegraphics[width=\columnwidth]{Philehalf40_0121_1659_1001.png}
    \caption{接触角$\frac{\pi}{12}$}
    \label{fig:phile_pi6}
  \end{subfigure}
  \begin{subfigure}{0.45\columnwidth}
    \centering
    \includegraphics[width=\columnwidth]{Philehalf40_0121_1659_1001_wl.png}
    \caption{補助線付き}
    \label{fig:phile_pi6wl}
  \end{subfigure}

  \begin{subfigure}{0.45\columnwidth}
    \centering
    \includegraphics[width=\columnwidth]{Philehalf40_0121_1652_1001.png}
    \caption{接触角$\frac{\pi}{4}$}
    \label{fig:phile_pi4}
  \end{subfigure}
  \begin{subfigure}{0.45\columnwidth}
    \centering
    \includegraphics[width=\columnwidth]{Philehalf40_0121_1652_1001_wl.png}
    \caption{補助線付き}
    \label{fig:phile_pi4wl}
  \end{subfigure}
  \begin{subfigure}{0.45\columnwidth}
    \centering
    \includegraphics[width=\columnwidth]{Philehalf40_0121_1704_1001.png}
    \caption{接触角$\frac{5}{12}\pi$}
    \label{fig:phile_pi3}
  \end{subfigure}
  \begin{subfigure}{0.45\columnwidth}
    \centering
    \includegraphics[width=\columnwidth]{Philehalf40_0121_1704_1001_wl.png}
    \caption{補助線付き}
    \label{fig:phile_pi3wl}
  \end{subfigure}

  \caption{接触角を変化させた親水性の床上の水滴}
\end{figure}



入力した接触角と得られた接触角の対応関係は、てんでんばらばらであった。図を見る限り入力した接触角の大小と出力された接触角の大小には相関関係があるようにも見えるが、それらを定量的に評価できる手続きを作り出すことができなかった。

以下の数値計算においては、接触角$\frac{\pi}{4},\frac{3}{4}\pi$をそれぞれ浸水性・疎水性の代表的な値として用い、それらの接触角と対応する実験での振る舞いと比較を行った。


\newpage
\subsection{剛体球の水面への衝突}
表面が親水性および疎水性の剛体球を水面に衝突させる。初速を変化させて、水面の変形による気泡の取り込みが見られるか否かを調べる。実験の値はDuez\cite{Duez2007}らによるものを参照した。

\subsubsection{実験の設定について}
数値計算で用いたパラメータや粒子の初期配置について説明する。

粒子の初期配置は図\ref{fig:initialStateWithRigid}である。60000ステップの計算を行い、流体部分が十分静止した後に剛体の運動を開始させた。

\begin{figure}[H]
  \centering
  \begin{minipage}{0.4\columnwidth}
    \centering
    \includegraphics[width=0.5\columnwidth]{initialStateWithRigid.png}
    \caption{計算の初期状態\label{fig:initialStateWithRigid} }
  \end{minipage}
    \begin{minipage}{0.5\columnwidth}
    \centering
    \includegraphics[width=0.6\columnwidth]{regionFull.png}
\caption{計算領域の全体図\label{fig:regionFull}}
  \end{minipage}
\end{figure}



計算領域全体の図は、一例として図\ref{fig:MLPhilevel80}の全体図を出力すると図\ref{fig:regionFull}のようになる。しかし、今回の数値計算は、剛体周辺における流体の振る舞いに注目しているため、計算領域全体の図は必要ではない。したがって、以下の紙面では剛体周辺を拡大したグラフを図として貼り付けている。



式(\ref{eqn:Tait})における係数Bの値について述べる。

密度の変化$\frac{\left| \delta \rho \right|}{\rho_0}\sim \frac{\left| v_f \right|^2}{c_s^2}\sim\eta$は1\%程度つまり$\eta\sim 0.01$が典型的な値として採用される\cite{Becker2007}。この誤差の範囲に収まるような音速$c_s$を設定する必要がある。

また、流体粒子間の相互作用の強さ$c_{\alpha\alpha}=60$とし、その他のパラメータは節\ref{subsec:surfaceCoef}と同じ値を用いた。

まず、節\ref{subsec:surfaceCoef}および節\ref{subsec:validity}と同じ$c_s$で計算を行い、剛体周辺での流体の速度を計算する。この計算で得られた流体の速度を基に、音速$c_s$を計算する。得られた流体の速度はおよそ40程度であったため、音速$c_s$は400程度が必要と判断した。
流体の速度を音速で割った値はMach数と呼ばれる。以下節\ref{subsec:surfaceCoef}および節\ref{subsec:validity}での音速の設定(Mach数:0.5)と、今述べた非圧縮性を満たす音速の設定(Mach数:0.1)、2通りの音速で剛体球の数値実験を行った。


空洞が生成されたか否かは、以下のような基準のもとで、目視に頼った。

水面に剛体が衝突すると、釣鐘を逆さにしたような凹みが生じ、その底部から剛体に向かって細長い空間が穿たれる。前者の大きな凹みを空間A、後者の細長い空間を空間Bと便宜的に呼ぶ。空間Bは、底の部分から左右の流体が合流することで空間が閉じていき、その合流部はやがて空間Aの底部に達する。しかし空間Bが底から閉じていく最中に、空間Bの上部でも左右の流体が合流し空間が閉じることがある。このとき空洞が生じたと判断する。
\subsubsection{Mach数大、疎水性の剛体}
\label{subsec:MLPhoby}
図\ref{fig:MachLPhoby}は、初速を変化させて疎水性の剛体を水面に衝突させ、左右に分離した流体が再び合体する箇所をグラフに出力したものである。
\begin{figure}[H]
  \centering
\begin{subfigure}{0.3\columnwidth}
  \centering
  \includegraphics[width=\columnwidth]{rec300Phobyball10_1221_1836_389_crop.png}
  \caption{衝突速度10}
  \label{fig:vel10}
\end{subfigure}
\begin{subfigure}{0.3\columnwidth}
  \centering
  \includegraphics[width=\columnwidth]{rec300Phobyball10_1221_2215_389_crop.png}
  \caption{衝突速度20}
  \label{fig:vel20}
\end{subfigure}
\begin{subfigure}{0.3\columnwidth}
  \centering
  \includegraphics[width=\columnwidth]{rec300Phobyball10_1222_0151_393_crop.png}
  \caption{衝突速度30}
  \label{fig:vel30}
\end{subfigure}
\begin{subfigure}{0.3\columnwidth}
  \centering
  \includegraphics[width=\columnwidth]{rec300Phobyball10_1222_0528_403_crop.png}
  \caption{衝突速度40}
  \label{fig:vel40}
\end{subfigure}
\begin{subfigure}{0.3\columnwidth}
  \centering
  \includegraphics[width=\columnwidth]{rec300Phobyball10_1222_0903_401_crop.png}
  \caption{衝突速度50}
  \label{fig:vel50}
\end{subfigure}
\begin{subfigure}{0.3\columnwidth}
  \centering
  \includegraphics[width=\columnwidth]{rec300Phobyball10_1222_1237_401_crop.png}
  \caption{衝突速度60}
  \label{fig:vel60}
\end{subfigure}
\end{figure}

\clearpage
\begin{figure}
\ContinuedFloat
  \begin{subfigure}{0.3\columnwidth}
  \centering
  \includegraphics[width=\columnwidth]{rec300Phobyball10_1222_1609_393_crop.png}
  \caption{衝突速度70}
  \label{fig:vel70}
\end{subfigure}
\begin{subfigure}{0.3\columnwidth}
  \centering
  \includegraphics[width=\columnwidth]{rec300Phobyball10_1222_1919_385_crop.png}
  \caption{衝突速度80}
  \label{fig:vel80}
\end{subfigure}
\caption{初速を変化させて疎水性の剛体を水面に衝突させた際の空洞の生成}
\label{fig:MachLPhoby}
\end{figure}
初速20の時点で、界面の変形による空洞の生成が確認できる。


\subsubsection{Mach数大、親水性の剛体}
図\ref{fig:hydrophilic}は、初速を変化させて親水性の剛体を水面に衝突させ、左右に分離した流体が再び合体する箇所をグラフに出力したものである。
\begin{figure}[H]
  \centering
\begin{subfigure}{0.3\columnwidth}
  \centering
  \includegraphics[width=\columnwidth]{rec300Phileball10_1219_2335_341_crop.png}
  \caption{衝突速度10}
  \label{fig:philevel10}
\end{subfigure}
\begin{subfigure}{0.3\columnwidth}
  \centering
  \includegraphics[width=\columnwidth]{rec300Phileball10_1220_0312_351_crop.png}
  \caption{衝突速度20}
  \label{fig:philevel20}
\end{subfigure}
\begin{subfigure}{0.3\columnwidth}
  \centering
  \includegraphics[width=\columnwidth]{rec300Phileball10_1220_0616_357_crop.png}
  \caption{衝突速度30}
  \label{fig:philevel30}
\end{subfigure}
\begin{subfigure}{0.3\columnwidth}
  \centering
  \includegraphics[width=\columnwidth]{rec300Phileball10_1220_0950_347_crop.png}
  \caption{衝突速度40}
  \label{fig:philevel40}
\end{subfigure}
\begin{subfigure}{0.3\columnwidth}
  \centering
  \includegraphics[width=\columnwidth]{rec300Phileball10_1220_1324_377_crop.png}
  \caption{衝突速度50}
  \label{fig:philevel50}
\end{subfigure}
\begin{subfigure}{0.3\columnwidth}
  \centering
  \includegraphics[width=\columnwidth]{rec300Phileball10_1220_1701_377_crop.png}
  \caption{衝突速度60}
  \label{fig:philevel60}
\end{subfigure}
\end{figure}
\clearpage
\begin{figure}
\ContinuedFloat
  \begin{subfigure}{0.3\columnwidth}
  \centering
  \includegraphics[width=\columnwidth]{rec300Phileball10_1220_2037_361_crop.png}
  \caption{衝突速度70}
  \label{fig:philevel70}
\end{subfigure}
\begin{subfigure}{0.3\columnwidth}
  \centering
  \includegraphics[width=\columnwidth]{rec300Phileball10_1221_0008_361_crop.png}
  \caption{衝突速度80}
  \label{fig:philevel80}
\end{subfigure}
\caption{初速を変化させて親水性の剛体を水面に衝突させた際の空洞の生成}
\label{fig:hydrophilic}
\end{figure}

\begin{figure}[H]
  \centering
  \includegraphics[width=0.8\columnwidth]{trajectory_x.png}
  \caption{重心x座標の時間発展}
  \label{fig:trajectory}
\end{figure}

衝突速度20の時点で空洞が生じていることが確認できる。


\subsubsection{Mach数小、疎水性の剛体}
流体の非圧縮性を保つために$c_s$を大きくし、疎水性の剛体を水面に衝突させた。
図を放り込む予定!
\begin{figure}[H]
  \centering
\begin{subfigure}{0.3\columnwidth}
  \centering
  \includegraphics[width=\columnwidth]{rec300Phobyball10_0111_1851_373_crop.png}
  \caption{衝突速度10}
  \label{fig:MSPhobyvel10}
\end{subfigure}
\begin{subfigure}{0.3\columnwidth}
  \centering
  \includegraphics[width=\columnwidth]{rec300Phobyball10_0111_2102_381_crop.png}
  \caption{衝突速度20}
  \label{fig:MSPhobyvel20}
\end{subfigure}
\begin{subfigure}{0.3\columnwidth}
  \centering
  \includegraphics[width=\columnwidth]{rec300Phobyball10_0111_2259_389_crop.png}
  \caption{衝突速度30}
  \label{fig:MSPhobyvel30}
\end{subfigure}
\begin{subfigure}{0.3\columnwidth}
  \centering
  \includegraphics[width=\columnwidth]{rec300Phobyball10_0112_0109_395_crop.png}
  \caption{衝突速度40}
  \label{fig:MSPhobyvel40}
\end{subfigure}
\begin{subfigure}{0.3\columnwidth}
  \centering
  \includegraphics[width=\columnwidth]{rec300Phobyball10_0112_0319_401_crop.png}
  \caption{衝突速度50}
  \label{fig:MSPhobyvel50}
\end{subfigure}
\begin{subfigure}{0.3\columnwidth}
  \centering
  \includegraphics[width=\columnwidth]{rec300Phobyball10_0118_1959_405_crop.png}
  \caption{衝突速度60}
  \label{fig:MSPhobyvel60}
\end{subfigure}
\end{figure}
\clearpage
\begin{figure}
\ContinuedFloat
  \begin{subfigure}{0.3\columnwidth}
  \centering
  \includegraphics[width=\columnwidth]{rec300Phobyball10_0119_2040_409_crop.png}
  \caption{衝突速度70}
  \label{fig:MSPhobyvel70}
\end{subfigure}
\begin{subfigure}{0.3\columnwidth}
  \centering
  \includegraphics[width=\columnwidth]{rec300Phobyball10_0118_2349_411_crop.png}
  \caption{衝突速度80}
  \label{fig:MSPhobyvel80}
\end{subfigure}
\caption{Mach数0.1、初速を変化させて疎水性の剛体を水面に衝突させた際の空洞の生成}
\label{fig:MachSPhoby}
\end{figure}

衝突速度80の時点で空洞が生じていると判断する。それ以下ではほとんど変化が見られない。
\subsubsection{Mach数小、親水性の剛体}
\label{subsec:MSPhile}
図\ref{fig:MachSPhile}は、Mach数$\sim0.1$の場合に、初速を変化させて親水性の剛体を水面に衝突させ、左右に分離した流体が再び合流する箇所をグラフに出力したものである。
\begin{figure}[H]
  \centering
\begin{subfigure}{0.3\columnwidth}
  \centering
  \includegraphics[width=\columnwidth]{rec300Phileball10_0102_2331_311_crop.png}
  \caption{衝突速度10}
  \label{fig:MLPhilevel10}
\end{subfigure}
\begin{subfigure}{0.3\columnwidth}
  \centering
  \includegraphics[width=\columnwidth]{rec300Phileball10_0103_0157_349_crop.png}
  \caption{衝突速度20}
  \label{fig:MLPhilevel20}
\end{subfigure}
\begin{subfigure}{0.3\columnwidth}
  \centering
  \includegraphics[width=\columnwidth]{rec300Phileball10_0103_0449_363_crop.png}
  \caption{衝突速度30}
  \label{fig:MLPhilevel30}
\end{subfigure}
\begin{subfigure}{0.3\columnwidth}
  \centering
  \includegraphics[width=\columnwidth]{rec300Phileball10_0103_0740_371_crop.png}
  \caption{衝突速度40}
  \label{fig:MLPhilevel40}
\end{subfigure}
\begin{subfigure}{0.3\columnwidth}
  \centering
  \includegraphics[width=\columnwidth]{rec300Phileball10_0103_1030_361_crop.png}
  \caption{衝突速度50}
  \label{fig:MLPhilevel50}
\end{subfigure}
\begin{subfigure}{0.3\columnwidth}
  \centering
  \includegraphics[width=\columnwidth]{rec300Phileball10_0103_1323_381_crop.png}
  \caption{衝突速度60}
  \label{fig:MLPhilevel60}
\end{subfigure}
\end{figure}
\clearpage
\begin{figure}
\ContinuedFloat
  \begin{subfigure}{0.3\columnwidth}
  \centering
  \includegraphics[width=\columnwidth]{rec300Phileball10_0103_1618_373_crop.png}
  \caption{衝突速度70}
  \label{fig:MLPhilevel70}
\end{subfigure}
\begin{subfigure}{0.3\columnwidth}
  \centering
  \includegraphics[width=\columnwidth]{rec300Phileball10_0103_1915_381_crop.png}
  \caption{衝突速度80}
  \label{fig:MLPhilevel80}
\end{subfigure}
\caption{Mach数0.1、初速を変化させて親水性の剛体を水面に衝突させた際の空洞の生成}
\label{fig:MachSPhile}
\end{figure}

速度80に於いて空洞が生じていることが分かる。速度70に於いて空洞が生じているか否かは少し微妙である。70と80の間が速度のしきい値であると判断するのが適当か。

\subsection{壁からの影響について}
\label{subsec:wall}
\begin{figure}[H]
  \centering
  \begin{minipage}{0.4\columnwidth}
    \centering
    \includegraphics[width=0.5\columnwidth]{rec600Phobyball10_0117_1713_201.png}
    \caption{剛体から側壁までの距離を長くとった計算の初期配置
      \label{fig:rec600} }
    \end{minipage}
    \begin{minipage}{0.5\columnwidth}
    \centering
    \includegraphics[width=0.6\columnwidth]{largeRegionFull.png}
    \caption{計算領域の全体図\label{fig:largeRegionFull}}
  \end{minipage}
\end{figure}

壁からの影響の有無を調べるため、図\ref{fig:rec600}のように計算領域の横幅を拡大した設定で計算を行った。図\ref{fig:largeRegionFull}のように側壁付近で流体は静止していることが分かる。
横幅を広くとった場合の計算に於いて、空洞の生成に関しては定性的、定量的にもほとんど変化は見られなかった。よって、空洞の生成に関して壁からの影響はそれほど影響が無く、節\ref{subsec:MLPhoby}から節\ref{subsec:MSPhile}における計算が十分妥当であるものとする。

\newpage
\section{議論}
\subsection{表面張力波について}
\label{subsec:discSurfWave}
表面張力を考慮しない重力波の計算において、数値計算で得られた周期と理論式から得られる周期はよく一致している。重力波の計算において、今回用いたSPH法は定量的にも信頼できる結果を出力しているものと考えられる。

表面張力波においては、$\gamma\sim 6000$という値が得られた。この値が水の物性値とどの程度対応しているのかを調べるため、次元解析を行う。
表面張力の次元は[$\rm{M/T^2}$]であるので、密度[$\rm{M/L^3}$]、長さ$\lambda$[L]、単位時間$T_g=(\frac{\lambda}{g})^{\frac{1}{2}}$[s]を用いて$\gamma\times T_g^2 \times \frac{1}{\rho} \times \frac{1}{\lambda^3}$という計算を行い、表面張力を無次元化する。
約18℃の水の場合、$\gamma\sim 7.3\times 10^{-2}\rm{[kgs^{-2}]}$、$\rho\sim 1.00\times 10^3$[$\rm{kgm^{-3}}$]、$\lambda\sim 1.7\times10^{-2}$[m]であるので\cite{tatsumiKiso}、無次元化された表面張力係数は
$ 2.6\times10^{-2}$という値が得られる。

数値計算の各値にも同様の計算を行う。$\gamma\sim6.4\times 10^3$、$\lambda\sim 6.0$などを用いると、$1.8\times 10^{-2}$という値が得られる。数値計算での$\gamma$と物性値の$\gamma$ではオーダーがほぼ同じとなっている。

よって、表面張力の大きさは、水の物性値と定量的によく一致しているものと考えられる。

今回の計算では、$c_{\alpha\alpha}$一例に対してのみ定量的な評価を行った。よって、他の$c_{\alpha\alpha}$や今回変化させることのなかった$k$の値を変えることで、異なる物性値の流体を再現できるかを調べることはできていない。

\subsection{接触角の問題について}
\label{subsec:discContact}
親水性領域に於いても$\frac{5}{12}\pi$では接触角が$\frac{\pi}{2}$を超える疎水性のような振る舞いが見られた。この原因について考察する。

今回用いた式(\ref{eqn:contactAngleSimplified})では$\theta>\frac{1}{2}\pi$以上では$c_{\omega\alpha}$が負の値を取り、流体部分が剛体から浮かぶように振る舞ってしまうことが確認される。Yang\cite{Yang2017}らはこの点に着目し、改良された接触角の計算式
\begin{equation}
  \label{eqn:modifiedContactAngle}
\cos \theta \approx 2c_{\omega\alpha}/c_{\alpha\alpha} -1
\end{equation}
を提案している。

式(\ref{eqn:modifiedContactAngle})においては、$c_{\omega\alpha}<\frac{1}{2}c_{\alpha\alpha}$の領域で$\theta>\frac{1}{2}\pi$となり疎水性の振る舞いを示すことになる。けれども、式(\ref{eqn:contactAngleSimplified})で$c_{\omega\alpha}<\frac{1}{2}c_{\alpha\alpha}$となるのは、$\theta>\frac{1}{3}\pi$の領域である。
そのため式(\ref{eqn:modifiedContactAngle})の方がより正しく濡れ性を再現している場合、式(\ref{eqn:contactAngleSimplified})を用いた今回の実験では$\theta>\frac{1}{3}\pi$の領域で疎水性のような振る舞いを示すことになったと考えられる。

ただし式(\ref{eqn:modifiedContactAngle})は$c_{\omega\alpha}$が負になることを防ぐために調整された”全く物理的に無意味な”式(not entirely physically meaningfulと表現されている)であり、この式を用いた計算がどれほど妥当なものであるかははっきりとした評価を下すことができず、今回の計算での採用は見送った。。

今回の計算で得られた結果は、(i)式(\ref{eqn:contactAngleSimplified})は物理的に正しい議論から導出されているものの、接触角の再現には定性的、定量的、ともに限界が存在している(ii)式(\ref{eqn:modifiedContactAngle})は接触角を定性的、定量的によりよく再現していると考えられるが、物理的な論拠が乏しく、数値実験内で用いることが適当か否かの問題が生じる、ということになる。

\subsection{空洞の生成について}
\label{subsec:discCavityForm}
節\ref{subsec:surfaceCoef}および節\ref{subsec:validity}と同じ$c_s$で計算を行い、得られた流体の速度から再計算した$c_s$で改めて数値実験を行った。Mach数を変えたことで計算結果に大きな差が生じていることから、本実験において、流体の非圧縮性は振る舞いに大きく影響していることが分かる。

まず、Mach数が大きい即ち圧縮性が顕著である場合での数値実験について考察する。
\begin{figure}[H]
  \centering
  \includegraphics[width=0.8\columnwidth]{thresholdcs80.png}
  \caption{Mach数が大きい場合での衝突速度のしきい値
    \label{fig:threscs80} }
\end{figure}

接触角と空洞が生成された速度を図にまとめると図\ref{fig:threscs80}のようになった。接触角は、設定した値ではなく、節\ref{subsec:validity}での計算結果から測定した値を用いた。

親水性の領域においてはかなり低い衝突速度にて空洞の生成が見られた。これは剛体が水面に衝突した際、非圧縮性が保たれていないために、水面を深く押し下げ、空洞を大きくしたためと考えられる。
疎水性の領域では空洞が見られたと言うこともできようが、今回の接触角のモデルで剛体と流体の間に空隙が存在しており、空洞生成を観測しようとする数値実験として適当であったか疑問が残る。

Mach数が小さい即ち流体の非圧縮性が十分に保たれている場合の数値実験について考察する。
\begin{figure}[H]
  \centering
  \includegraphics[width=0.8\columnwidth]{thresholdcs400.png}
  \caption{Mach数が小さい場合での衝突速度のしきい値
    \label{fig:threscs400} }
\end{figure}

接触角と空洞が生成された速度を図にまとめると図\ref{fig:threscs400}のようになった。
親水性・疎水性の場合ともに速度が80の時に空洞の生成が見られた。この80という速度に対して次元解析を施してみる。
\subsection{次元解析}
\label{subsec:label}
数値計算の粘度$\mu$は、式(\ref{eqn:kinArtVisc})にある$\alpha h c_s$の部分に相当すると考えられる。各値を代入すると$0.02\cdot 0.1 \cdot 400 \sim 8.00\times 10^{-1}$が得られる。レイノルズ数は$Re=\frac{\rho U D}{\mu}$なので、今回の数値計算における$Re$は$\frac{10^3\cdot U \cdot 1}{8.00\times10^{-1}}\sim \frac{U}{8}\times10^4$になる。今$U$は10から80の間であるため、$Re\sim 10^4-10^5$となる。この値はDuezらの実験での$Re$と同じ範囲にあたる。

次に慣性項と表面張力項の比であるWeber数$We=\frac{\rho U^2 D}{\gamma}$を比較してみよう。各値を代入すると$\frac{10^3\cdot U^2 \cdot 1}{6.0\times10^3}\sim \frac{U^2}{6.0}$となる。$U$の範囲を考えると$We\sim 10^1 -10^2$となる。Duezらの実験では$We\sim10^2-10^3$程度であるのでオーダーが$10^1$ほど異なる。



\newpage
\section{結論}
\subsection{まとめ}
Duezの実験と比較して、剛体の衝突速度を増加させることで空洞生成が見られるようになるという点では定性的に一致している。速度のしきい値は、実験においては親水性でのしきい値のほうが疎水性でのしきい値より大きくなっているが、この大小関係は今回の数値計算に置いて見られなかった。
疎水性の領域に関して、今回用いた接触角のモデルでは剛体と流体の間に空隙が存在しており、空洞生成を観測しようとする数値実験ではかなりマズイ。

\subsection{今後の展望}
疎水性において、剛体部分と流体部分の間に空隙が生じるという非物理的な現象が確認されている。これを改善したモデルを用いる方法が第一である。けれどもそのモデルがどれほど物理的に正しいものであるかは…
また今回の計算は2次元で行っている。
左右の壁の位置が近すぎたのでは?Yang\cite{Yang2017}にある改良された接触角のモデルではどうか?
SPHの誤差を修正するモデルを加えてはどうか?そのモデルの物理的な妥当性は議論が必要で云々……など今回試していない事柄について書き記す予定。
\newpage

\section{謝辞}
本研究を進めるにあたり、藤定義准教授には研究に対するアドバイスや議論をしていただきました。また、松本剛助教授には、日々の数値計算や解析、修論の構成および推敲に関して多大なご助力を戴きました。博士課程の蛭田佳樹さん、岸達郎さんには、物理や数学の知識、また計算機についてなど様々なことでお世話になりました。修士課程の丸石崇史さん、山田遥さん、酒井亮さんには、研究生活を通じて多くをお世話になりました。ここに感謝の意を表します。
\newpage

\bibliography{thesis}
\bibliographystyle{junsrt}

\end{document}