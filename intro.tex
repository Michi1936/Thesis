\documentclass[]{jsarticle}
\usepackage[dvipdfmx]{graphicx}

\begin{document}
\section{イントロダクション}
\subsection{界面の変形}
水面に水滴あるいは剛体を衝突させた際の現象即ち界面の変形や気泡の生成は身近な物理現象の一つであり、古くから研究が行われてきた。

界面の変形に関する研究では、Worthingtonによるもの \cite{worthington1908}がその嚆矢として擧げられることが多い。彼は水面に水滴を衝突させる実験を複数回行い、その度に少しずつ異なるタイミングでストロボを焚いて写真を撮影することで、水飛沫の生成の連続写真を作成することに成功した。以来、撮影器具の進歩に合わせ、より鮮明な画像が撮影され続けてきた。水しぶきの画像は、高速度撮影の代表的な成果の一つであり、1957年Harold Edgertonが撮影したミルククラウンの写真は、TIMEが選んだ”最も影響ある写真100枚”に選ばれている。

界面が変形した際に空気が取り込まれる現象が確認されている。

空気が取り込まれる現象は音生成の原因となっている。Minnaert\cite{minnaert1933xvi}が”小川のせせらぎ、瀑布の轟音、海の潮騒についてほとんど分かっていない”と述べたこの現象は、Franz\cite{Franz1959}やPumphrey\cite{Pumphrey1990}らによって研究され、Leightonによる浩瀚な研究書\it{The Acoustic Bubble}\cite{leightonacoustic}にまとめられた。近年では、ハイスピードカメラを用いて、取り込まれた泡が実際に振動している様子が撮影されている\cite{Phillips2018}。

\subsection{SPH法}
流体を粒子の集合として計算する手法は粒子法と呼ばれ、\rm{Smoothed Particle Hydrodynamics}(略称SPH)法は粒子法のモデルの一つである。

SPH法は1977年に、宇宙物理の問題を解くための手法として、MonaghanとGingold\cite{Gingold1977}のペア及びLucy\cite{Lucy1977}が別々に発表した。1980年台には宇宙物理に加えて、電磁流体力学や弾性・破壊過程の計算に用いられるようになったが、日常的な流体に適応されるようになったのは1990年代の半ばになってからだ。1994年にMonaghan\cite{Monaghan1994}が発表した手法が現在流体計算に用いられているモデルの基本となり、以来計算の目的に応じて様々な改良が施され続けている。

粒子法は流体を物理量を持った粒子の集合として計算する手法であり、計算領域を格子に分割して計算する手法との差を強調してmesh-freeな手法であると説明されることもある。
\label{subsec:label}


\bibliography{thesis}
\bibliographystyle{junsrt}


\end{document}
