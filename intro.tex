\documentclass[]{jsarticle}
\usepackage[dvipdfmx]{graphicx}

\begin{document}
\section{イントロダクション}
\subsection{界面の変形}
水面に水滴あるいは剛体を衝突させた際の界面の変形は最も身近な物理現象の一つであり、古くから研究が行われてきた。

界面の変形は非常に短い時間スケールの話であるため、この現象を捉えるための研究が数多く行われている。最も古いのはWorthingtonによるものである \cite{worthington1908}。彼は水面に水滴を衝突させる実験を複数回行い、その度に少しずつ異なるタイミングでストロボを焚いて写真を撮影することで、水飛沫の生成の連続写真を作成することに成功した。以来、撮影器具の進歩に合わせ、より鮮明な画像が撮影され続けてきた。水しぶきの画像は、高速度撮影の代表的な成果として代表的なものの一つであり、1957年Harold Edgertonが撮影したミルククラウンの写真は、TIMEが選んだ”最も影響ある写真100枚”に選ばれている。

界面が変形した際に空気が取り込まれる現象が確認されている。

空気が取り込まれる現象は音生成の原因となっている。Minnaert\cite{minnaert1933xvi}が”小川のせせらぎ、瀑布の轟音、海の潮騒についてほとんど分かっていない”と述べたこの現象は、Franz\cite{Franz1959}やPumphrey\cite{Pumphrey1990}らによって研究され、Leightonによる浩瀚な研究書\it{The Acoustic Bubble}\cite{leightonacoustic}にまとめられている。近年では、ハイスピードカメラを用いて、取り込まれた泡が実際に振動している様子が撮影されている\cite{Phillips2018}。

\bibliography{thesis}
\bibliographystyle{junsrt}


\end{document}
